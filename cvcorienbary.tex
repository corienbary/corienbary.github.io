\documentclass[a4paper,11pt]{article}
\usepackage[dutch,english]{babel}

\title{\textbf{Corien Bary}}
\author{\Large{\textsc{curriculum vitae}}}
\date{}

\usepackage{longtable}

\usepackage[english]{babel}
\hyphenation{Nij-me-gen}
\usepackage{hyperref}

\begin{document}
\maketitle


\noindent  Corien Liesbeth Anke Bary\\
%Gemsstraat 5\\
%6531 TC Nijmegen\\
%The Netherlands\\
corien.bary@ru.nl\\
\url{corienbary.github.io}\\
%2 children\\
Year of birth: 1981\\

\section*{Intro}

I am a full professor in Semantics at Radboud University Nijmegen. In my research I address bigger questions about language and communication, and I'm always keen to develop new ways to tackle older and newer issues. I am eager to learn new techniques and to bring together researchers from different fields. As such, my research combines insights and techniques from a range of disciplines. In my Veni project on speech reports and my ERC Starting Grant project on perspective in language, I combined the fields of semantics and pragmatics, philosophy of language, classical scholarship on Ancient Greek, computational linguistics, experimental linguistics and narratology. In my current research I investigate how we undertake, attribute and avoid commitments through language. As part of this, I am currently leading a proof-of-method study to investigate the potential of fEMG (facial electromyography, the recording of facial muscle activity) as a window into this aspect of communication which forms the backbone of our social lives.\\\vspace{1em} 

\noindent Research topics:

\begin{itemize}
\item commitments in communication\vspace{-0.5em}
\item perspective in language and narratology\vspace{-0.5em}
\item indexicality\vspace{-0.5em}
\item evidentiality\vspace{-0.5em}
\item speech and attitude reports\vspace{-0.5em}
\item tense\vspace{-0.5em}
\item aspect\vspace{-0.5em}
\item performativity\vspace{-0.5em}
\end{itemize}




\section*{Employment}

September 2020 - present. \emph{Coordinator of the Center for Cognition, Culture and Language}, Radboud University Nijmegen.\\\\
September 2019 - present. \emph{Chair of the Department Philosophy of Mind and Language}. Faculty of Philosophy, Theology and Religious Studies, Radboud University Nijmegen.\\\\
March 2017 - present. \emph{Professor in Logical Semantics}.  Department Philosophy of Mind and Language. Faculty of Philosophy, Theology and Religious Studies, Radboud University Nijmegen.\\\\
November 2013 - February 2012. \emph{Associate Professor} (UHD).  Department Philosophy of Mind and Language. Faculty of Philosophy, Theology and Religious Studies, Radboud University Nijmegen. ERC Starting Grant project: \emph{Unraveling the Language of Perspective}.\\\\
April 2011 - October 2013. \emph{Assistant Professor} (UD, tenure). Department Philosophy of Mind and Language. Faculty of Philosophy, Theology and Religious Studies, Radboud University Nijmegen.  NWO Veni project: \emph{Reports from the past} until August 2013.\\\\
January 2010 - March 2011.  \emph{Researcher}. Faculty of Philosophy, Radboud University Nijmegen.
NWO Veni project: \emph{Reports from the past}.\\\\
September 2009 - December 2009. \emph{Assistant Professor} (UD, non-tenure). Faculty of Philosophy, Radboud University Nijmegen.\\\\ 
August 2008 - August 2009. \emph{Researcher and teacher}. Faculty of Philosophy and Faculty of Arts, Radboud University Nijmegen.\\\\
August 2004 - July 2008. \emph{Junior researcher}. Department of
Philosophy of Language and Logic, Radboud University Nijmegen.\\\\
%October 2003 - December 2003. \emph{Student assistant} for
%Department of Philosophy of
%Language and Logic, University of Nijmegen.\\\\
%April 2003 - June 2003. \emph{Student assistant} for the Department of Philosophy of Language and Logic, University of
%Nijmegen.\\\\
%September 2001 - April 2004. \emph{Student assistant} for the
%Department of Ancient and Medieval Philosophy, University of
%Nijmegen. Editorial assistant for the series \emph{Medieval and
%Early Modern Science} (Brill, Leiden).\\\\
%January 2001 - September 2001. \emph{Student assistant} for the project
%\emph{Corpus Gesproken Nederlands} (Spoken Dutch Corpus),
%University of Nijmegen and Max Planck Institut f\"{u}r
%Psycholinguistik Nijmegen.\\


\section*{Extended stays abroad}

%\begin{longtable}{p{160pt} p{210pt}}
August - October 2016. University of British Columbia, Vancouver, financed by an ERC starting grant.\\\\
June - August 2013. University of Oslo (Norway), financed by a Veni grant.\\\\
April - August 2011.  Humboldt-Universit\"at zu Berlin (Germany), sponsored by the Niels Stensen Stichting.\\\\
December 2009 - February 2010. University of Oslo (Norway), financed by a Rubicon grant\\\\
June - July 2008.  Universit\"at Stuttgart (Germany), financed by a Frye stipend\\\\
Augustus - December 2007. University of Texas at Austin (US), financed by a Fulbright grant\\\\
January - April 2005. University of Edinburgh (Scotland), financed by a VSB grant\\


\pagebreak

\section*{Education}
% \begin{tabular}{p{60pt}p{280pt} } March 2009. & Ph.D. in Philosophy (\emph{cum laude}).  Radboud University Nijmegen.\\ &Thesis: \emph{Aspect in Ancient Greek. A semantic analysis of the aorist and imperfective.}\\ & Promotor: Rob van der Sandt.\\ & Members of the manuscript committee: Andr\'e Lardinois, Nick Asher, Alexander Mourelatos.\\ & Members of the \emph{cum laude} committee: + Hans Kamp, Barbara Partee. \\
%\end{tabular}\\\\
March 2009.  Ph.D. in Philosophy (\emph{cum laude}),  Radboud University Nijmegen. Thesis: \emph{Aspect in Ancient Greek. A semantic analysis of the aorist and imperfective.} Promotor: Rob van der Sandt. Members of the manuscript committee: Andr\'e Lardinois, Nick Asher, Alexander Mourelatos. Members of the \emph{cum laude} committee: + Hans Kamp, Barbara Partee.
\\\\
March 2006. ``Bovenbouw''\emph{Philosophy} (\emph{cum laude}). Equivalent to the English Bachelor's plus Master's degree.
Radboud University Nijmegen.\\\\
September 2004. Doctoral \emph{Ancient Greek and Latin Language and Culture} (\emph{cum laude}).  Equivalent to the English Bachelor's plus Master's degree. Radboud University Nijmegen.\\\\
July 2000. Propedeuse \emph{Ancient Greek and Latin Language and
Culture} (\emph{cum laude}). University of Nijmegen.\\\\June 1999.
\emph{Gymnasium} diploma (\emph{cum laude}). Dominicus
College, Nijmegen.\\


\section*{Grants and honours}

%2005. Third price in the ESSLLI Student Session.\\\\
2013. ERC Starting Grant for the project \emph{Unraveling the Language of Perspective} (2013-2019).\\\\
2012. Grant of the NWO Incentive Fund Open Access.\\\\
2012. Jubileumprijs of the Netwerk Vrouwelijke Hoogleraren at the Radboud University Nijmegen.\\\\
2009.  NWO Veni grant for the research project \emph{Reports from the past} at the Radboud University Nijmegen (2010-2013).\\\\
2009.  Stipend of the Niels Stensen Stichting for a stay at the Humboldt-Universit\"at zu Berlin.\\\\
2009. NWO Rubicon grant for a stay at the University of Oslo.\\\\
2007. Fryestipendium, grant to promote women in academics. Radboud University Nijmegen.\\\\ 
2007. Best MA thesis Award. Faculty of Philosophy, Radboud University Nijmegen.\\\\  
2006. Fulbright grant for a stay at the University of Texas at Austin.\\\\ 
2004. VSB grant for a stay at the University of
Edinburgh.\\


\section*{Publications}
Hariet Yates, Corien Bary, Peter de Swart and Bob van Tiel (to appear), fEMG as a window into conversational commitments. \emph{Proceedings of Sinn und Bedeutung 29}.\\\\ 
Corien Bary (2025), Speech acts, common ground and commitments. \emph{Linguistics and Philosophy}. https://doi.org/10.1007/s10988-025-09434-y \\\\
Leopold Hess, Corien Bary and Bob van Tiel (2024), Commitments de lingua and assertoric commitments: the case of expressives. \emph{Proceedings of Semantics and Linguistic Theory} 33.\\\\
Corien Bary (2023), Speech reports and evidence. In: Corien Bary and Natasha Korotkova (2023), Special issue on Expressing Evidence. \emph{Journal of Pragmatics}.\\\\
Corien Bary, Leopold Hess and Bob van Tiel (2023), Expressives do shift, but not on their own. \emph{Proceedings of the 23rd Amsterdam Colloquium}, pp. 31--37.\\\\
Corien Bary and Natasha Korotkova (eds) (2022-3), Special Issue on Expressing Evidence. \emph{Journal of Pragmatics}.\\\\
Corien Bary (2022), Present tense. In: Daniel Altshuler (ed) \emph{Linguistics Meets Philosophy}. CUP.\\\\
Corien Bary and Emar Maier (2021), The landscape of speech reporting. \emph{Semantics and Pragmatics} 14(8).\\\\
Corien Bary, Leopold Hess and Kees Thijs (eds) (2020), Special issue on The Language of Perspective. \emph{Open Library of Humanities}.\\\\
Leopold Hess and Corien Bary (2019), Narrator language and character language in Thucydides: a quantitative study of narrative perspective. \emph{Digital Scholarship in the Humanities}.\\\\
Corien Bary, Daniel Altshuler, Kristen Syrett and Peter de Swart (2018), Factors licensing embedded present tense in speech reports. In: Uli Sauerland and Stephanie Solt (eds.), \emph{Proceedings of Sinn und Bedeutung 22}, pp. 127--142\\\\
Corien Bary (2018), Reporting someone else's speech: the use of the optative and accusative-and-infinitive as reportative markers in Herodotus' \emph{Histories}. Open Library of Humanities 4(1): 20, pp. 1--38.\\\\
Corien Bary (2018). Counting events. \emph{Philosophical Inquiry} (special issue in honour of Alexander Mourelatos).\\\\
Corien Bary, Peter Berck, and Iris Hendrickx (2017). A memory-based lemmatizer for Ancient Greek. \emph{Proceedings of DATeCH 2017 (Digital Access to Textual Cultural Heritage)}, pp. 91--95..\\\\
Corien Bary (2017). Reportative markers in Ancient Greek. In: Felicia Logozzo and Paolo Poccetti (eds.): Ancient Greek Linguistics: new approaches, insights, perspectives, pp. 292-302. Berlin: De Gruyter.\\\\
Corien Bary, Leopold Hess, Kees Thijs, Peter Berck and Iris Hendrickx (2017). Annotating speech, attitude and perception reports. \emph{Proceedings of the EACL  (European Chapter of the Association for Computational Linguistics) Conference}.\\\\
Emar Maier and Corien Bary (2015). Three puzzles about negation in non-canonical speech reports. In: Thomas Brochhagen, Floris Roelofsen and Nadine Theiler (eds.) \emph{Proceedings of the 2015 Amsterdam Colloquium.}, pp.\ 246-255.\\\\
Corien Bary and Daniel Altshuler (2015). Double access. In: Eva Csipak and Hedde Zeijlstra (eds.), \emph{Proceedings of Sinn und Bedeutung 19}, pp. 89-106.\\\\
Corien Bary and Emar Maier (2014), Unembedded Indirect Discourse. In: Urtzi Etxeberria, Anamaria Falaus, Aritz Irurtzun and Bryan Leferman (eds.), \emph{Proceedings of Sinn und Bedeutung 18}, pp. 77-94.\\\\
Corien Bary (2012), The Ancient Greek tragic aorist revisited. \emph{Glotta} 88(4), 31--53.\\\\
Corien Bary and Markus Egg (2012), Variety in Ancient Greek aspect interpretation. \emph{Linguistics and Philosophy} 35(2), 111--134.\\\\
Corien Bary (2012), Tense in Ancient Greek reports. \emph{Journal of Greek Linguistics} 12(1), 29-50.\\\\
Corien Bary and Rob van der Sandt (eds) (2012), \emph{Journal of Greek Linguistics} 12(1). Special issue on Ancient Greek Linguistics.\\\\
Corien Bary and Dag Haug (2011), Temporal anaphora across and inside sentences: the function of participles. \emph{Semantics and Pragmatics} 4(8), 1-56.\\\\
Corien Bary and Dag Haug (2011), Inter- and intrasentential anaphora: the case of the Ancient Greek participle. In: Neil Ashton, Anca Chereches, and David Lutz (eds), \emph{Proceedings of SALT 2011}, pp. 373-392.\\\\
Corien Bary (2010), Aspect in Ancient Greek. A semantic analysis of the aorist and imperfective. Dissertation announcement. \emph{Mnemosyne}, 64, 700-701.\\\\
Corien Bary (2009). The perfective/imperfective distinction: coercion or aspectual operators? In: Lotte Hogeweg, Helen de Hoop, and Andrej Malchukov (eds.), \emph{Cross-linguistic semantics of Tense, Aspect and Modality}, pp. 33-53. Amsterdam \& Philadelphia: Benjamins.\\\\ 
Corien Bary and Emar Maier 2009. The dynamics of tense under attitudes: Anaphoricity and \emph{de se} interpretation in the backward shift past. In: Yasuo Nakayama et al. (eds.), \emph{New Frontiers in Artificial Intelligence}, Lecture Notes in Computer Science. Berlin/Heidelberg: Springer.\\\\ 
Corien Bary and Emar Maier (2008). The dynamics of tense under attitudes: Anaphoricity and \emph{de se} interpretation in the backward shift past. In: Norry Ogata (ed.), \emph{Proceedings of the Fifth International Workshop on Logic and Engineering of Natural Language Semantics}, pp.\ 103-117. Asahikawa: JSAI.\\\\
Corien Bary and Markus Egg (2007). Aspect and coercion in Ancient Greek. In: Maria Aloni, Paul Dekker and Floris 
Roelofsen (eds.), \emph{Proceedings of the sixteenth Amsterdam Colloquium}, pp.\ 49-54.  Amsterdam: ILLC/Department of Philosophy 
University of Amsterdam.\\\\
Corien Bary and Peter de Swart (2005). Arguments against
arguments. Additional accusatives in Latin and Ancient Greek. In:
Judith Gervain (ed.), \emph{Proceedings of the Tenth ESSLLI
Student
Session}, pp. 12-24.\\\\
Emar Maier, Corien Bary \& Janneke Huitink (eds) (2005).
\emph{Proceedings of Sinn und Bedeutung 9}. NCS, Nijmegen.\\


\section*{Presentations}

November 12--14 2025. Keynote at the Syntax and Semantics Conference in Paris (CSSP 2025).\\\\
September 17--19 2025.  \emph{Evidentiality and Speaker Commitment: A Facial EMG Study}.  XPRAG 2025. Cambridge.
May 22 2025. \emph{fEMG as a window into conversational commitments: validating the method and further applications}. Semantics Colloquium. Frankfurt (with Harriet Yates, Bob van Tiel and Peter de Swart).\\\\
May 10 2025. \emph{Commitments in communicatie}. Wijsgerig Festival DRIFT. Amsterdam.\\\\ 
January 20 2025. \emph{Narrative perspecitve through vocabulary choice: a case study on Thucydides} (based on joint work with Leopold Hess).\\\\
September 17--19 2024. \emph{fEMG as a window into conversational commitments.} Sinn und Bedeutung 29. Noto (with Harriet Yates, Peter de Swart and Bob van Tiel).\\\\
September 5--7 2024. \emph{fEMG as a window into conversational commitments.} AMLaP 30. Edinburgh (with Harriet Yates, Peter de Swart and Bob van Tiel).\\\\
August 20--21 2024. \emph{Speech reports, commitments and evidence}. Workshop Epistemological and Linguistic Perspectives on Testimony. Stockholm.\\\\
May 12--14 2023. \emph{Commitments de lingua and assertoric commitments: the case of expressives}. SALT at Yale. New Haven, CT (with Leopold Hess and Bob van Tiel).\\\\
March 24 2023. \emph{Present tense: challenges for a semantic theory}. DIP Colloquium, Amsterdam.\\\\
December 19-21 2022. \emph{Expressives do shift but not on their own}. Amsterdam Colloquium, Amsterdam (with Leopold Hess and Bob van Tiel).\\\\ 
September 29--30 2022. \emph{Perspective Taking through Content Words}. Workshop Fiction and Narrative across Media. Groningen.\\\\
June 23--25 2022. \emph{Present under Past}. Presentation at the workshop The Dynamics of Semantics: Past, Present and Future Ways of Thinking about Meaning, in honor of Hans Kamp's 80th birthday.\\\\
March 31 -- April 2 2021. \emph{The Landscape of Speech Reporting}. Workshop on the Semantics and Pragmatics of Clause-Embedding Predicates. Columbus, Ohio.\\\\ 
June 6-8 2019. \emph{Speech reports and evidence}. Keynote lecture at the workshop Expressing Evidence. Konstanz.\\\\
January 23 2019. \emph{A multidisciplinary approach to narrative perspective: combining semantics, philosophy of language, computational linguistics and classics}. Presentation at the workshop Multiple Perspectives, Nijmegen (with Leopold Hess).\\\\
June 15 2018. Inaugural lecture \emph{Change perspective}. Nijmegen.\\\\ 
April 13 2018. \emph{Narrator language and character language in Thucydides}. Conference A corpus and usage-based approach to Ancient Greek: from the Archaic period until the Koine. Riga (with Leopold Hess).\\\\
November 15 2017. \emph{Present tense in speech reports: tracking other people's beliefs}. Theoretical Linguistics Seminar. Oslo (based on joint work with Daniel Altshuler, Kristen Syrett and Peter de Swart).\\\\
September 7-9 2017. \emph{Factors licensing embedded present tense in speech reports}. Sinn und Bedeutung 2017. Berlin/Potsdam (with Daniel Altshuler, Kristen Syrett and Peter de Swart).\\\\
June 21-23 2017. \emph{Factors licensing embedded present tense in speech reports}. Xprag 2017 Conference. Cologne (with Daniel Altshuler, Kristen Syrett and Peter de Swart).\\\\
June 21-23 2017. \emph{Factors licensing embedded present tense in speech reports}. XPrag Conference, Cologne (with Daniel Altshuler, Kristen Syrett and Peter de Swart).\\\\
June 1 2017. \emph{A memory-based lemmatizer for Ancient Greek}. DATeCH (Digital Access to Textual Cultural Heritage) 2017 conference, G\"ottingen (with Iris Hendrickx and Peter Berck).\\\\
May 10 2017. \emph{At-issue events and non-at-issue evidence in the semantics of speech reports}. UiO-Xprag.de workshop on Non-at-issue Meaning and Information Structure, Oslo (with Emar Maier).\\\\
April 3 2017.  \emph{Annotating speech, attitude, and perception reports}. Linguistic Annotation Workshop at the EACL, Valencia (with Kees Thijs, Leopold Hess, Peter Berck and Iris Hendrickx).\\\\
30 September 2016. \emph{Eventive and evidential speech reports}. UBC Linguistics Colloquium Series, Vancouver. \\\\
8 April 2016. \emph{Why the Historical Present is not the Mirror Image of Free Indirect Discourse}. Perspectivization Workshop at the 39th GLOW, G\"ottingen.\\\\
15 January 2016. \emph{Backgrounding and commitment in parenthetical reports and reportative evidentials}. Workshop Backgrounded Reports:
Commitment and Negation in Parenthetical Reports and Reportative Evidentials, Nijmegen (with Emar Maier).\\\\
17 December 2015. \emph{Three puzzles about negation in non-canonical speech reports.} The 2015 Amsterdam Colloquium, Amsterdam (with Emar Maier).\\\\
24 March 2015. \emph{Indirectness marking in Ancient Greek}. International Conference on Ancient Greek Linguistics, Rome (with Emar Maier).\\\\
3 March 2015. \emph{Perspective-taking in person and time}. Workshop on Perspective-taking at the DGfS. Leipzig.\\\\ 
24 October 2014. \emph{Indirectness marking in Herodotus}. Conference Textual Strategies in Greek and Latin War Narrative. Amsterdam.\\\\
15 September 2014. \emph{Double access and acquaintance}. Conference Sinn und Bedeutung 19. G\"ottingen (with Daniel Altshuler).\\\\
27 June 2014. \emph{Double access and acquaintance}. Conference Semantics and Philosophy in Europe. Berlin (with Daniel Altshuler).\\\\
23 November 2013.Reaction to Sander Orriens' presentation 'What do you imply, sir? Perfect Predicaments in Ancient Greek.' Biennial meeting of Greek and Latin linguists of the Low Lands. Katwijk.\\\\
12 October 2013. \emph{Unembedded indirect discourse}. Investigating Semantics: Experimental and Philosophical Approaches. Bochum (with Emar Maier).\\\\ 
11 September 2013. \emph{Unembedded indirect discourse}. Sinn und Bedeutung 18. Vitoria-Gasteiz (with Emar Maier). \\\\
4 June 2013. \emph{Ancient Greek aspect}. Advanced seminar Speaking of Events (guest lecture in a course taught by Daniel Altshuler and Hana Filip). Heinrich-Heine-Universit\"at D\"usseldorf.\\\\ 
24 April 2013. \emph{Free Indirect Discourse in Ancient Greek?} Workshop Perspectives on Narrative, Nijmegen.\\\\
21 January 2013. \emph{Free Indirect Discourse in Ancient Greek?} Workshop The Language of Reports, Nijmegen.\\\\
27 May 2011. \emph{Participles}. Workshop for Rob van der Sandt's 65th birthday, Nijmegen.\\\\
21 May 2011. \emph{Inter- and intrasentential anaphora: the case of the Ancient Greek participle}. Poster presentation at Semantics and Linguistic Theory (SALT) 21, Rutgers University, New Brunswick (with Dag Haug).\\\\
8 March, 2011. \emph{Tense in Ancient Greek reports}. Semantics and Pragmatics Colloquium, Nijmegen.\\\\
4 February 2011. \emph{Tense and aspect in Ancient Greek}. Seminar of the Amsterdam Center for Language and Communication, Amsterdam.\\\\
17 December 2010. \emph{Reports in Ancient Greek}. Conference Ancient Greek and Semantic Theory, Nijmegen.\\\\ 
17 February 2010. \emph{The temporal interpretation of participles in Ancient Greek}.
Seminar in theoretical linguistics. University of Oslo (with Dag Haug).\\\\
8 December 2009. \emph{The logics of Ancient Greek tense and aspect}. Logic and language seminar, Tilburg.\\\\
28 May 2009. \emph{Aspect in Ancient Greek.} Leiden Utrecht Semantic Happenings, Leiden.\\\\
1 October 2008. \emph{Anaphoricity vs. de se interpretation: the case of backward shifted past.} Sinn und Bedeutung 13. Stuttgart (with Emar Maier).\\\\ 
9 June 2008. \emph{The dynamics of tense under attitudes: Anaphoricity and de se interpretation in the backward shift past.} Fifth International Workshop on Logic and Engineering of Natural Language Semantics (LENLS), Asahikawa, Japan (with Emar Maier).\\\\
3 April 2008. \emph{Anaphoricity vs. de se interpretation: the case of backward shifted past.} Conference on Semantics and Modelisation, Toulouse (with Emar Maier).\\\\
28 February 2008. \emph{Aspect and coercion in Ancient Greek.} Jahrestagung der Deutschen Gesellschaft f\"ur Sprachwissenschaft, Bamberg (with Markus Egg).\\\\
20 December 2007. \emph{Aspect and coercion in Ancient Greek.} Ereignis-Semantik Workshop, T\"ubingen (with Markus Egg).\\\\
18 December 2007. \emph{Aspect and coercion in Ancient Greek.} Amsterdam Colloquium (with Markus Egg).\\\\ 
30 November 2007. \emph{Een nieuwe kijk op de tragische aoristus}. Antieke Wereld-dag, Nijmegen.\\\\
10 November 2007. \emph{Aspectuele herinterpretatie}. Biennial meeting of Greek and Latin linguists, Katwijk (with Markus Egg).\\\\
29 June 2007. \emph{Aspect and the temporal structure of discourse}. 6th International Colloquium on Ancient Greek Linguistics, Groningen.\\\\
6 December 2006. \emph{Aspect in Ancient Greek}. Syntax and Semantics Colloquium, Department of Linguistics, University of Texas at Austin.\\\\
16 November 2006. \emph{The perfective/imperfective distinction: coercion or aspectual operators?}. Workshop TAMTAM: Crosslinguistic semantics of tense, aspect, modality, Nijmegen.\\\\
26 June 2006. \emph{The tragic aorist in Ancient Greek as a speech act phenomemon}. Semantics in the Netherlands Day, Utrecht.\\\\
2 June 2006. \emph{The tragic aorist in Ancient Greek as a speech act phenomemon}. TABU Day, Groningen.\\\\
22 May 2006. \emph{The tragic aorist in Ancient Greek}. Workshop for Rob van der Sandt's 60th birthday, Nijmegen.\\\\
12 May 2006. \emph{The Imperfect and Aorist in Ancient Greek}. DIP Colloquium, Amsterdam.\\\\
6 December 2005. \emph{Aspect in Ancient Greek}. Semantics Colloquium, Nijmegen.\\\\
15 August 2005. \emph{Arguments against arguments. Additional
accusatives in Latin and Ancient Greek}. Student Session at the
European Summer School of Logic, Language and Information,
Edinburgh (with Peter de Swart).\\\\
25 November 2004. \emph{Additional objects in Latin and Ancient
Greek}. Semantics in the Netherlands Day, Nijmegen (with Peter de Swart).\\\\
3 March 2003. \emph{Ancient Greek Monsters}. Szklarska
Poreba Workshop 4, Poland (with Emar Maier).\\\\
7 November 2002. \emph{Ancient Greek Monsters}. Semantics in the Netherlands Day, Nijmegen (with Emar
Maier).\\

%\end{longtable}


\section*{Supervision of PhD students}

Harriet Yates (ongoing). Radboud University Nijmegen.\\\\  
Kees Thijs (2021), \emph{Polysemous particles in Ancient Greek. A study with special reference to \emph{m\`en} and \emph{d\`e}.} Radboud University Nijmegen.\\\\
Leopold Hess (2019), \emph{Perspective and word choice: a study of expressives, slurs, and narrative.} Radboud University Nijmegen.\\\\
Per Erik Solberg (2017), \emph{The discourse-semantics of long-distance reflexives}. University of Oslo (with Dag Haug).








\section*{Teaching}

Since 2005, I have taught over 60 courses on logic, philosophy of language, Ancient Greek linguistics, argumentation theory, language and society, and philosophy of mind for students of Philosophy, AI, Classics and Psychology and Philosophy, Politics and Society at Radboud University Nijmegen.

%Fall 2019. \emph{The Power of Words} for students of Philosophy, Politics and Society.\\\\
%Spring 2019. \emph{Philosophy of Language} for students of Philosophy.\\\\
%Spring 2018. \emph{Philosophy of Language} for students of Philosophy.\\\\
%Spring 2018. \emph{Argumentation Theory} for students of Philosophy.\\\\
%Spring 2017. \emph{Philosophy of Language} for students of Philosophy.\\\\
%Spring 2017. \emph{Bachelor thesis course on bivalence} for students of Philosophy.\\\\
%Spring 2016. \emph{Philosophy of Language} for students of Philosophy.\\\\
%Spring 2015. \emph{Philosophy of Language} for students of Philosophy.\\\\
%Spring 2015. \emph{Ancient Greek Linguistics} for students of Classics.\\\\
%Fall 2013. \emph{How to make your case in semantics and pragmatics?} Class in the Nunspeet seminar (OIKOS) for PhD and research master students in classics.\\\\
%Fall 2013. \emph{Philosophy of Language} for students of Philosophy.\\\\
%Spring 2013. \emph{Philosophy of Language} for students of Philosophy\\\\
%Spring 2013. \emph{Introduction to Logic} for students of Philosophy.\\\\
%Fall 2011. \emph{Philosophy of Language} for students of Philosophy.\\\\
%Fall 2011. \emph{Philosophy of Mind} for students of Psychology, tutorials.\\\\
%Fall 2011. \emph{Ancient Philosophy} for students of Classics.\\\\
%Fall 2010. \emph{Ancient Philosophy} for students of Classics.\\\\
%Fall 2010. \emph{Ancient Philosophy} for students of Philosophy.\\\\
%Fall 2010. \emph{Philosophy of language} for students of Philosophy.\\\\
%Fall 2009. \emph{Philosophy of Mind} for students of Psychology, tutorials.\\\\
%Fall 2009. \emph{Argumentation Theory} for students of
%Philosophy.\\\\
%Fall 2009. \emph{Ancient Philosophy} for students of Classics.\\\\
%Spring 2009. \emph{Ancient Philosophy} for students of Classics.\\\\
%Spring 2009. \emph{Philosophy of Language} for students of Philosophy.\\\\
%Fall 2008. \emph{Ancient Philosophy} for students of Philosophy.\\\\
%Fall 2008. \emph{Ancient Philosophy} for students of Classics.\\\\
%Fall 2007. \emph{Argumentation Theory} for students of
%Philosophy.\\\\
%Spring 2007. \emph{Philosophy of Language} for students of Philosophy.\\\\
%Spring 2007. \emph{Introduction to Logic} for students of Philosophy.\\\\
%Spring 2006. \emph{Introduction to Logic} for students of Philosophy.\\\\
%Fall 2005. \emph{Argumentation Theory} for students of
%Philosophy.\\\\
%Fall 2004. \emph{Introduction to Logic} for students of Philosophy.
%\\\\ 
%Fall 2004. \emph{Introduction to Logic} for students of Artificial
%Intelligence.\\






\section*{Academic services% and memberships
}

\subsection*{Organization}
2019. Co-organizer of the workshop Multiple perspectives, Radboud University Nijmegen, January 23 (with Leopold Hess and Kees Thijs).\\\\
2017. Co-organizer of the workshop Particles and the dimensions of meaning, Radboud University Nijmegen, June 29-30 (with Kees Thijs).\\\\
2006 - present. Co-organizer of the Semantics \& Pragmatics Colloquium (until 2009 Semantics Colloquium), Radboud University Nijmegen.\\\\
2016. Co-organizer of the workshop Backgrounded Reports: Commitment and Negation in Parenthetical Reports and Reportative Evidentials, Radboud University Nijmegen, January 15 (with Emar Maier).\\\\
2014. Co-organizer of the workshop Language and the Law. Radboud University Nijmegen, December 15.\\\\
2013. Organizer of the workshop The Language of Reports. Radboud University Nijmegen, January 21--22.\\\\
2010. Co-organizer of the conference Ancient Greek and Semantic Theory, Radboud University Nijmegen, December 16-18 (with Rob van der Sandt).\\\\
 2004. Local co-organizer of Sinn und Bedeuting 9. Radboud University Nijmegen.


\subsection*{Reviewing}

2011 - present. Member of the Editorial Board of \emph{Journal of Semantics}.\\\\
Reviewer for  \emph{Natural Language and Linguistic Theory}, \emph{Mnemosyne}, \emph{Linguistics and Philosophy}, \emph{The Philosophical Quarterly}, \emph{Journal of Greek Linguistics}, \emph{Lingua}, \emph{Glossa},  \emph{Open Linguistics}, ESSLLI Student Session, Sinn und Bedeutung, SALT, Formal Diachronic Semantics, Philosophers Rally, GLOW, Xprag events, Springer books, Wiley Blackwell, DFG, NWO, Israel Science Foundation, NIAS fellowships.



\subsection*{Committees}
2000 - present. Member of the Research Management Team of the Faculty of Philosophy, Theology, and Religious Studies.\\\\
2018. Member of the NWO Open Competition Panel Linguistics.\\\\
2013 - 2018. Chair of the Library Committee of the Faculty of Philosophy, Theology, and Religious Studies (member since 2011).\\\\%2018 is geschat
2015 - 2017. Member of the onderdeelcommissie of the Faculty of Philosophy, Theology, and Religious Studies.\\\\
2012 - 2015. Member of the opleidingscommissie of Philosophy of the Faculty of Philosophy, Theology, and Religious Studies.\\\\
 2005 -  2007. Representative of the Faculty of
Philosophy in the PON (Promovendi Overleg Nijmegen).\\\\
2000 - 2002. Member of student representation, Classical Studies. University of Nijmegen.\\\\
%September 1999 - August 2000. Member of the social program
%committee of the Sodalicium Classicum Noviomagense, Student
%Association of Classical Studies, University of Nijmegen.\\



\end{document}
